\tableofcontents
\section{AIModule}
\subsection{接口类}
\begin{codebox}[公共成员函数]
AIModule()
virtual void onEnd (bool isWinner)
virtual void onFrame ()
virtual void onNukeDetect(Position target)
virtual void onPlayerLeft(Player player)
virtual void onReceiveText(Player player, std::string text)
virtual void onSaveGame(std::string gameName)
virtual void onSendText(std::string text)
virtual void onStart()
virtual void onUnitComplete(Unit unit)
virtual void onUnitCreate(Unit unit)
virtual void onUnitDestroy(Unit unit)
virtual void onUnitDiscover(Unit unit)
virtual void onUnitEvade(Unit unit)
virtual void onUnitHide(Unit unit)
virtual void onUnitMorph(Unit unit)
virtual void onUnitRenegade(Unit unit)
virtual void onUnitShow(Unit unit)
\end{codebox}

\subsubsection{详细描述}
AIModule是一个虚拟类,旨在由自定义AI类实现或继承。

如果BWAPI调用了任何这些预定义的接口功能,则保证初始化Broodwar接口

\begin{warning}
    在调用这些函数的线程以外的任何线程中使用BWAPI都会产生意外行为,并可能崩溃您的Bot。只要所有BWAPI交互都仅限于调用线程,多线程AI是有可能实现的。
    
\end{warning}

\begin{note}
    replay被认为是game,并调用与标准game相同的所有callbacks。
\end{note}
\subsubsection{构造函数和析构函数文档}
\begin{codebox}[构造函数\&析构函数]
    BWAPI::AIModule::AIModule ()
\end{codebox}
\subsubsection{成员函数文档}
\begin{tcolorbox}[colback=white, colframe=black!60!white, title=onStart(), arc=0mm]
    \begin{minted}[frame=none]{cpp}
    virtual void BWAPI::AIModule::onStart()
    \end{minted}
    此函数仅在游戏开始时调用一次。  
    它旨在让 AI 模块在此函数中进行任何数据初始化。
    \begin{warning}
        在调用此函数之前使用 Broodwar 接口可能会导致未定义行为,并使你的 AIbot 崩溃。(例如,在类的静态初始化期间调用接口可能会导致问题。)
    \end{warning}
\end{tcolorbox}
    
\begin{tcolorbox}[colback=white, colframe=black!60!white, title=onEnd(), arc=0mm]
\begin{minted}[frame=none]{cpp}
virtual void BWAPI::AIModule::onEnd(bool isWinner)
\end{minted}
只在游戏结束时被调用一次。
\begin{parameter}
    \begin{itemize}
        \item \texttt{isWinner}:决定当前 player 是否赢得本场比赛的 bool 变量,在获胜时值为 true,在失败或游戏是 replay 时值为 false
    \end{itemize}
\end{parameter}
\end{tcolorbox}

\begin{tcolorbox}[colback=white, colframe=black!60!white, title=onFrame(), arc=0mm]
    \begin{minted}[frame=none]{cpp}
    virtual void BWAPI::AIModule::onFrame()
    \end{minted}
    在每个逻辑帧中被调用一次。  
    用户将大部分代码放在这个函数中。
\end{tcolorbox}
    

\begin{tcolorbox}[colback=white, colframe=black!60!white, title=onSendText(), arc=0mm]
\begin{minted}[frame=none]{cpp}
virtual void BWAPI::AIModule::onSendText(std::string text)
\end{minted}
在 user 尝试发送消息时被调用。  
这个函数让 bot 能够执行 user 输入的文本指令用以 debug。
\begin{parameter}
    \begin{itemize}
        \item \texttt{text}:String 对象,包含 user 发送的具体文本信息
    \end{itemize}
\end{parameter}
\begin{note}
    如果 Flag::UserInput 被禁用,该函数不会被调用。
\end{note}
\end{tcolorbox}

\begin{tcolorbox}[colback=white, colframe=black!60!white, title=onReceiveText(), arc=0mm]
    \begin{minted}[frame=none]{cpp}
    virtual void BWAPI::AIModule::onReceiveText(Player player, std::string text)
    \end{minted}
    在 client 接收到来自另一位 player 的消息时被调用。  
    该函数能用于在团队游戏中从队友处检索信息,或只是为了回应其他 player。
    \begin{parameter}
        \begin{itemize}
            \item \texttt{player}:Player 接口对象代表文本消息的所有者
            \item \texttt{text}:player 发送的文本信息
        \end{itemize}
    \end{parameter}
    \begin{note}
        当前 player 发送的信息不会调用此函数。
    \end{note}
\end{tcolorbox}
    
\begin{tcolorbox}[colback=white, colframe=black!60!white, title=onPlayerLeft(), arc=0mm]
\begin{minted}[frame=none]{cpp}
virtual void BWAPI::AIModule::onPlayerLeft(Player player)
\end{minted}
当一个 player 离开时被调用。  
他们的所有 unit 将会自动分配给 neutral player,并保留他们的 colour 和 alliance 参数。
\begin{parameter}
    \begin{itemize}
        \item \texttt{player}:表示离开游戏的 player 的 Player 接口对象
    \end{itemize}
\end{parameter}
\end{tcolorbox}

\begin{tcolorbox}[colback=white, colframe=black!60!white, title=onNukeDetect(), arc=0mm]
    \begin{minted}[frame=none]{cpp}
    virtual void BWAPI::AIModule::onNukeDetect(Position target)
    \end{minted}
    当 Nuke 在地图上的某个位置启动时调用。
    \begin{parameter}
        \begin{itemize}
            \item \texttt{target}:包含 Nuke 目标位置的 Position 对象。如果目标位置不可见且 Flag::CompleteInformation 已禁用,则目标将为 Position::Unknown
        \end{itemize}
    \end{parameter}
\end{tcolorbox}
    
\begin{tcolorbox}[colback=white, colframe=black!60!white, title=onUnitDiscover(), arc=0mm]
\begin{minted}[frame=none]{cpp}
virtual void BWAPI::AIModule::onUnitDiscover(Unit unit)
\end{minted}
当一个 Unit 变为可访问时调用。
\begin{parameter}
    \begin{itemize}
        \item \texttt{unit}:Unit 接口对象,表示刚刚变为可访问的 unit
    \end{itemize}
\end{parameter}
\begin{note}
    此函数包含 Flag::CompleteMapInformation 状态
\end{note}
\begin{refer}
    \begin{itemize}
        \item \texttt{BWAPI::AIModule::onUnitShow}
    \end{itemize}
\end{refer}
\end{tcolorbox}

\begin{tcolorbox}[colback=white, colframe=black!60!white, title=onUnitEvade(), arc=0mm]
    \begin{minted}[frame=none]{cpp}
    virtual void BWAPI::AIModule::onUnitEvade(Unit unit)
    \end{minted}
    当一个 Unit 变为不可访问时调用。
    \begin{parameter}
        \begin{itemize}
            \item \texttt{unit}:Unit 接口对象,表示刚刚变为不可访问的 unit
        \end{itemize}
    \end{parameter}
    \begin{note}
        此函数包含 Flag::CompleteMapInformation 状态
    \end{note}
    \begin{refer}
        \begin{itemize}
            \item \texttt{BWAPI::AIModule::onUnitHide}
        \end{itemize}
    \end{refer}
\end{tcolorbox}
    
\begin{tcolorbox}[colback=white, colframe=black!60!white, title=onUnitShow(), arc=0mm]
\begin{minted}[frame=none]{cpp}
virtual void BWAPI::AIModule::onUnitShow(Unit unit)
\end{minted}
当以前不可见的 Unit 变为可见时调用。
\begin{parameter}
    \begin{itemize}
        \item \texttt{unit}:Unit 接口对象,表示刚刚变为可见的 unit
    \end{itemize}
\end{parameter}
\begin{note}
    此函数包含 Flag::CompleteMapInformation 状态
\end{note}
\begin{refer}
    \begin{itemize}
        \item \texttt{BWAPI::AIModule::onUnitDiscover}
    \end{itemize}
\end{refer}
\end{tcolorbox}

\begin{tcolorbox}[colback=white, colframe=black!60!white, title=onUnitHide(), arc=0mm]
\begin{minted}[frame=none]{cpp}
virtual void BWAPI::AIModule::onUnitHide(Unit unit)
\end{minted}
当以前可见的 Unit 变为不可见时调用。
\begin{parameter}
    \begin{itemize}
        \item \texttt{unit}:Unit 接口对象,表示即将超出范围的 unit
    \end{itemize}
\end{parameter}
\begin{note}
    此函数包含 Flag::CompleteMapInformation 状态
\end{note}
\begin{refer}
    \begin{itemize}
        \item \texttt{BWAPI::AIModule::onUnitEvade}
    \end{itemize}
\end{refer}
\end{tcolorbox}
\begin{tcolorbox}[colback=white, colframe=black!60!white, title=onUnitCreate(), arc=0mm]
    \begin{minted}[frame=none]{cpp}
    virtual void BWAPI::AIModule::onUnitCreate(Unit unit)
    \end{minted}
    在创建任何 unit 时调用。
    \begin{parameter}
        \begin{itemize}
            \item \texttt{unit}:Unit 接口对象,表示刚刚创建的 unit
        \end{itemize}
    \end{parameter}
    \begin{note}
        由于 Broodwar 的内部机制,此函数排除了 Zerg 的变形(morphing)以及在瓦斯矿(Vespene Geyser)上建造建筑的情况。
    \end{note}
    \begin{refer}
        \begin{itemize}
            \item \texttt{BWAPI::AIModule::onUnitMorph}
        \end{itemize}
    \end{refer}
\end{tcolorbox}

\begin{tcolorbox}[colback=white, colframe=black!60!white, title=onUnitDestroy(), arc=0mm]
\begin{minted}[frame=none]{cpp}
virtual void BWAPI::AIModule::onUnitDestroy(Unit unit)
\end{minted}
当一个 unit 因死亡或其他原因从游戏中移除时被调用。
\begin{parameter}
    \begin{itemize}
        \item \texttt{unit}:Unit 接口对象,表示刚刚被摧毁或以其他方式完全从游戏中移除的 unit
    \end{itemize}
\end{parameter}
\begin{note}
    当一个工蜂(Drone)变形为萃取器(Extractor)时,工蜂会被从游戏中移除,而瓦斯矿(Vespene Geyser)会变形为萃取器。  
    如果一个 unit 是可见的,并且被摧毁,那么在调用此事件之前,会先调用 \texttt{onUnitHide}。
\end{note}
\end{tcolorbox}

\begin{tcolorbox}[colback=white, colframe=black!60!white, title=onUnitMorph(), arc=0mm]
    \begin{minted}[frame=none]{cpp}
    virtual void BWAPI::AIModule::onUnitMorph(Unit unit)
    \end{minted}
    当一个 unit 的 UnitType 发生变化时被调用。  
    例如,当一个工蜂(Drone)变形为孵化场(Hatchery)、攻城坦克(Siege Tank)进入攻城模式(Siege Mode),或者一个瓦斯矿(Vespene Geyser)被建造了精炼厂(Refinery)时。
    \begin{parameter}
        \begin{itemize}
            \item \texttt{unit}:Unit 接口对象,表示 UnitType 的改变的 unit
        \end{itemize}
    \end{parameter}
    \begin{note}
        如果 UnitType 变为或从 \texttt{UnitTypes::Unknown} 变化,则不会触发此事件。
    \end{note}
\end{tcolorbox}
    
\begin{tcolorbox}[colback=white, colframe=black!60!white, title=onUnitRenegade(), arc=0mm]
\begin{minted}[frame=none]{cpp}
virtual void BWAPI::AIModule::onUnitRenegade(Unit unit)
\end{minted}
当一个 unit 的所有权发生变化时被调用。  
这种情况发生在使用神族的“心灵控制”(Mind Control)技能时,或者在“使用地图设置”(Use Map Settings)中单位的所有权发生变化时。
\begin{parameter}
    \begin{itemize}
        \item \texttt{unit}:Unit 接口对象,表示改变所有权的 unit
    \end{itemize}
\end{parameter}
\end{tcolorbox}
\begin{tcolorbox}[colback=white, colframe=black!60!white, title=onSaveGame(), arc=0mm]
    \begin{minted}[frame=none]{cpp}
    virtual void BWAPI::AIModule::onSaveGame(std::string gameName)
    \end{minted}
    当 Broodwar 游戏状态被保存到文件时被调用。
    \begin{parameter}
        \begin{itemize}
            \item \texttt{gameName}:包含游戏被保存为的文件名的 String 对象
        \end{itemize}
    \end{parameter}
\end{tcolorbox}

\begin{tcolorbox}[colback=white, colframe=black!60!white, title=onUnitComplete(), arc=0mm]
\begin{minted}[frame=none]{cpp}
virtual void BWAPI::AIModule::onUnitComplete(Unit unit)
\end{minted}
当 unit 的状态从 'incomplete' 变为 'complete' 时被调用。
\begin{parameter}
    \begin{itemize}
        \item \texttt{unit}:Unit 对象,表示刚刚完成训练或建造的 Unit
    \end{itemize}
\end{parameter}
\end{tcolorbox}
